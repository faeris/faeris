\documentclass[a4paper,10pt]{article}
\usepackage{CJK}
\usepackage{indentfirst}
\usepackage{graphicx}
\usepackage{bibentry,natbib}
\usepackage{fancyhdr}
\usepackage{lastpage}

\usepackage[top=2.54cm, bottom=2.54cm, left=3.18cm, right=3.18cm]{geometry}
\begin{document}
\begin{CJK}{UTF8}{song}
\begin{center}
\Large 向量(vector)与矩阵(matrix)
\end{center}
\section{向量}
\subsection{向量的定义}
在数学中,向量的定义如下:把即有大小又有方向的量叫作向量。由n个数组成的有序数组($a_{1},a_{2},....a_{n}$ )叫做一个n维向量,其中 $a_i$ (i=1,2,3,..,n) 叫作它的第i个分量。向量有两种表示方法,横向和纵向,通常把:
\begin{quote}
$a=(a_1,a_2,...,a_n)$叫做行向量,把$\left( \begin{array}{c} a_1 \\ a_2 \\ \vdots{} \\ a_n \end{array} \right)$叫做列向量
\end{quote}
\subsection{向量的运算}
设向量$\alpha=(\alpha{}_{1},\alpha{}_{2},\ldots{},\alpha{}_{n})$ , $\beta=(\beta{}_{1},\beta{}_{2},\ldots{},\beta{}_{n})$。
\subsubsection{基本运算}
% \item 相等:当且仅当 $\alpha{}_{i}=\beta{}_{i} (i=1,2,\ldots{},n)$ 时,$\alpha=\beta$。
向量的加运算在向量所对应的分量上进行,对于给定的两个向量P和Q,可以将加运算P+Q定义为:
\begin{quote}
$P+Q=(P_{1}+Q_{1},P_{2}+Q_{2},\ldots{},P_{n}+Q_{n}) $
\end{quote}
两个向量的减运算表示为P-Q,实际上可以用简单的加运算来表示:P+(-Q)。

标量k与向量Q相乘:
\begin{quote}
$kQ=(kQ_{1},kQ_{2},\ldots{},kQ_{n})$
\end{quote}
根据以上的定义,可以得到向量运算的一些基本性质。
\begin{description}
 \item[定理:]对于给定的任何两个系数a和b,以及任何三个向量P、Q和R,存在下述运算规律:
\begin{enumerate}
\item $P+Q=Q+P$
\item $(P+Q)+R=P+(Q+R)$
\item $(ab)P=a(bP)$
\item $a(P+Q)=aP+aQ$
\item $(a+b)Q=aP+bP$
\end{enumerate}
\end{description}

\subsubsection{模与规格化}
一个n维向量V的模是一个系数,记作$|V|$,由下面公式得到:
\begin{quote}
$|V|=\sqrt{\sum_{i=1}^{n}V_{i}^{2}}$
\end{quote}
展开为:
\begin{quote}
$|V|=\sqrt{V_{1}^{2}+V_{2}^{2}+\ldots{}+V_{n}^{2}} $
\end{quote}
向量的模有时又称为向量的范数或长度,模为1的向量称作单位长度向量,或者简称为单位向量。

如果向量V至少有一个非零的分量,则可以对该向量进行规格化:
\begin{quote}
$V'=\frac{V}{|V|}$
\end{quote}
\begin{description}
\item[定理:]对于任意给定的系数a,以及任意的两个向量P与Q,有以下性质:
\begin{enumerate}
\item $|P|\ge{}0$
\item 当且仅当$P=(0,0,\cdots{},0)$时,$|P|=0$
\item $|aP|=|a||P|$
\item $|P+Q|\le{}|P|+|Q|$
\end{enumerate}
\end{description}


\subsubsection{点积}
两个向量的每个对应分量的乘积之和称为点积,有时又称做向量间的数量积或内积,两个n维向量P和Q的点积记做$P\cdot{}Q$,其结果由下面公式给出,即:
\begin{quote}
$P\cdot{}Q=\sum_{i=1}^{n}P_{i}\cdot{}Q_{i}
$
\end{quote}
展开得:
\begin{quote}
$P\cdot{}Q=P_{1}\cdot{}Q_{1}+P_{2}\cdot{}Q_{2}+\ldots{}+P_{n}\cdot{}Q_{n}$
\end{quote}
点积$P\cdot{}Q$也可以用矩阵的乘积形式给出,即:
\begin{quote}
$PQ^{T}=\left[
\begin{array}{cccc}
 P_{1} & P_{2} &  \cdots{} & P_{n} 
\end{array}\right]
\left[
\begin{array}{c}
Q_{1} \\
Q_{2} \\
\vdots{} \\
Q_{n}
\end{array}\right]
$
\end{quote}
在几何空间中,点积可以直观的定义为(其中$\theta$表示两个向个向量这间的角度):
\begin{quote}
 $P\cdot{}Q=|P||Q|cos\theta $
\end{quote}
如果给定两个向量,它们之间的夹角可以通过下列公式得到:
\begin{quote}
$cos\theta=\frac{P\cdot{}Q}{|P||Q|}$
\end{quote}
\begin{description}
\item[定理:]对于给定的系数a和任意三个向量P、Q和R,存在以下性质:
\begin{enumerate}
\item $P\cdot{}Q=Q\cdot{}P$
\item $(aP)\cdot{}Q=a(P\cdot{}Q)$
\item $P\cdot{}(Q+R)=P\cdot{}Q+P\cdot{}R$
\item $P\cdot{}P=|P|^2 $
\item $|P\cdot{}Q|\le{}|P||Q|$
\end{enumerate}

\end{description}
\subsubsection{投影}
在许多情况下,需要将一个向量P分解为平行于和垂直于另一个向量Q的分量,这个时候需要用到投影,下面的公式表示向量P到Q的投影,记作$proj_{Q}P$:
\begin{displaymath}
proj_{Q}P=\frac{(P\cdot{}Q)}{|Q|^2}Q
\end{displaymath}
向量P相对于向量Q的垂直分量记作$perp_{Q}P$,即用向量P减去平行分量即可:
\begin{displaymath}
perp_{Q}P=P-proj_{Q}P=P-\frac{P\cdot{}Q}{Q^2}Q
\end{displaymath}
P到Q的投影是一个线性变换的过程,可以将该过程表示为矩阵向量的乘积。在三维情况下,$proj_{Q}P$也可以用下面的公式来计算,即:
\begin{quote}
$proj_{Q}P=\frac{1}{|Q|^2}\left[
\begin{array}{ccc}
Q_{x}^2 &Q_{x}Q_{y} & Q_{x}Q_{z} \\
Q_{x}Q_{y} & Q_{y}^2 & Q_{y}Q_{z} \\
Q_{x}Q_{z} & Q_{y}Q_{z} & Q_{z}^2 \\
\end{array} \right]
\left[
\begin{array}{c}
P_{x} \\
P_{y} \\
P_{z} \\
\end{array}
\right]
$
\end{quote}



\subsubsection{叉积}
两个3D向量P和Q的叉积记作$P\times{}Q$,其结果向量为:
\begin{displaymath}
P\times{}Q=(P_yQ_z-P_zQ_y,P_zQ_x-P_xQ_z,P_xQ_y-P_yQ_z)
\end{displaymath}
叉积也可以写成向量矩阵乘积的形式:
\begin{quote}
\begin{math}
P\times{}Q=
\left[\begin{array}{ccc}
0 & -P_{z}  & P_{y} \\
P_{z} & 0 & -P_{x} \\
-P_{y} & P_{x} & 0 \\
\end{array} \right]
\left[\begin{array}{c}
Q_{x} \\
Q_{y} \\
Q_{z} \\ 
\end{array}\right]
\end{math}
\end{quote}
叉积的模与P和Q之间的夹角a关系为:
\begin{quote}
$|P\times{}Q|=|P||Q|sin(a)$
\end{quote}
向量u与向量v的另一种算法为,建立这样一个矩阵: 
\begin{quote}
\begin{math}
P \times Q 
\quad\Rightarrow\quad
\left| \begin{array}{ccc}
	i & j & k \\
	P_{x} & P_{y} & P_{z} \\
	Q_{x} & Q_{y} & Q_{z} \\
       \end{array}
\right|  \\
\end{math}
\end{quote}
其中i,j,k分别是与x轴,y轴,z轴平行的单位矢量。
\begin{description}
\item[定理:]对于给定的两个系数a和b和任意三个3D向量P、Q和R,有以下性质存在
\begin{enumerate}
\item $P\times{}Q=-(Q\times{}P)$
\item $(aP)\times{}Q=a(P\times{}Q)$
\item $P\times{}(Q+R)=P\times{}Q+P\times{}R$
\item $P\times{}P=0=(0,0,0)$
\item $(P\times{}Q)\cdot{}R=(R\times{}P)\cdot{}Q=(Q\times{}R)\cdot{}P $
\end{enumerate}
\end{description}

\subsection{向量的线性组合}
\begin{description}
\item[定义:]m个向量$V_1,V_2,\cdots{},V_m$的线性组合具有如下形式的向量:
\begin{displaymath}
w=a_1V_1+a_2V_2+\cdots{}+a_mV_m
\end{displaymath}
其中$a_1,a_2,\cdots{},a_m$是标量
\end{description}
\subsubsection{向量的仿射组合}
如果线性组合系数$a_1,a_2,\cdots{},a_m$的和等于1,那么它就是仿射组合,因此上面等式中的线性组合若满足:
\begin{displaymath}
a_1+a_2+\cdots{}+a_m=1
\end{displaymath}
就是一个仿射组合,如果每一个系数都还满足非负数,则该线性组合为一个凸组合





\section{矩阵}
\subsection{矩阵的定义}
矩阵是一个矩形阵列,有指定的行和列,通常说矩阵为$m\times n$,表示它有m行和n列,表示为:
\begin{quote}
\begin{math}
A=\left|
\begin{array}{cccc}
a_{11} & a_{12} & \cdots & a_{1m} \\
a_{21} & a_{22} & \cdots & a_{2m} \\
\vdots & \vdots & \ddots & \vdots \\
a_{n1} & a_{n2} & \cdots & a_{nm} \\
\end{array}
   \right|
\end{math}
\end{quote}
在3D图形学中,常用的矩阵为$2\times 2$与$3\times 3$矩阵。
\subsection{矩阵的运算}
设$m\times n$矩阵$
A=\left|
\begin{array}{cccc}
a_{11} & a_{12} & \cdots & a_{1m} \\
a_{21} & a_{22} & \cdots & a_{2m} \\
\vdots & \vdots & \ddots & \vdots \\
a_{n1} & a_{n2} & \cdots & a_{nm} \\
\end{array}
   \right|
$
与矩阵$
B=\left|
\begin{array}{cccc}
b_{11} & b_{12} & \cdots & b_{1m} \\
b_{21} & b_{22} & \cdots & b_{2m} \\
\vdots & \vdots & \ddots & \vdots \\
b_{n1} & b_{n2} & \cdots & b_{nm} \\
\end{array}
   \right|
$

\begin{enumerate}
 \item 矩阵A与B加法与减法:
\begin{quote}
\begin{math}
A\pm B=\left|
\begin{array}{cccc}
a_{11} \pm b_{11}& a_{12}\pm b_{11}& \cdots & a_{1m} \pm b_{1m} \\
a_{21} \pm b_{21}& a_{22}\pm b_{22}& \cdots & a_{2m} \pm b_{2m} \\
\vdots & \vdots & \ddots & \vdots \\
a_{n1}\pm b_{n1} & a_{n2}\pm b_{n2} & \cdots & a_{nm}\pm b_{nm}\\
\end{array}
\right|
\end{math}
\end{quote}
\item 矩阵的转置,将矩阵的行转换为列,例如将矩阵A转置后,记为$A^{T}$:
\begin{quote}
\begin{math}
A^{T}=
\left|
\begin{array}{cccc}
a_{11} & a_{21} & \cdots & a_{n1} \\
a_{12} & a_{22} & \cdots & a_{n2} \\
\vdots & \vdots & \ddots & \vdots \\
a_{1m} & a_{2m} & \cdots & a_{nm} \\
\end{array}
\right|
\end{math}
\end{quote}
\item 矩阵A与标量k的乘法:
\begin{quote}
\begin{math}
A*k=k*A=k*\left|
\begin{array}{cccc}
a_{11} & a_{12} & \cdots & a_{1m} \\
a_{21} & a_{22} & \cdots & a_{2m} \\
\vdots & \vdots & \ddots & \vdots \\
a_{n1} & a_{n2} & \cdots & a_{nm} \\
\end{array}
   \right| 
=
\left|
\begin{array}{cccc}
k*a_{11} & k*a_{12} & \cdots & k*a_{1m} \\
k*a_{21} & k*a_{22} & \cdots & k*a_{2m} \\
\vdots & \vdots & \ddots & \vdots \\
k*a_{n1} & k*a_{n2} & \cdots & k*a_{nm} \\
\end{array}
   \right|
\end{math}
\end{quote}
\item 矩阵的乘法,假设$m\times n$的矩阵A与矩阵$n\times v$的矩阵B相乘得$m\times v$的矩阵C,其中:
\begin{quote}
\begin{math}
c_{ij}=A_{i1}B_{1J}+A_{i2}B_{2j}+\cdots{}+A_{in}B_{nj}=\sum_{r=1}^{n}{A_{ir}B_{rj}}
\end{math}
\end{quote}
\item 矩阵的运算定律,假没$m\times n$矩阵A、B和C,以及标量k
\begin{enumerate}
\item $A+B=B+A$(加法交换律)
\item $A+(B+C)=(A+B)+C$(加法结合律)
\item $A*(B*C)=(A*B)*C$(乘法结合律)
\item $A*(B+C)=A*B+A*C$(分配律)
\item $k*(A+B)=k*A+k*B$(分配律)
\end{enumerate}
\end{enumerate}

\subsection{应用矩阵的变化}
\subsubsection{平移}
在3D空间中执行平移,需要将点P(x,y,z)平移到新的位置$P^{'}$(x+dx,y+dy,z+dz),用$4\times 4$执行钜阵变化为:
\begin{quote}
\begin{math}
M_{t}=\left|
\begin{array}{cccc}
1&0&0&dx \\
0&1&0&dy \\
0&0&1&dz \\
0&0&0&1 \\
\end{array}
\right|
\end{math} 
\end{quote}
给定p=[x y z 1],执行如下平移变换:
\begin{quote} 
\begin{math}
p^{'}=M_{t}*p= \left|
\begin{array}{cccc}
1&0&0&dx \\
0&1&0&dy \\
0&0&1&dz \\
0&0&0&1 \\
\end{array}
\right|*
\left[\begin{array}{c}
x  \\
y \\
z  \\
1 
\end{array}\right]
=\left[
\begin{array}{c} 
(x+1*dx) \\
(y+1*dx) \\
 (z+1*dz) \\
 1 
\end{array} \right]
=\left[ \begin{array}{c} (x+dx) \\ (y+dx) \\ (z+dz) \\ 1  \end{array} \right]
\end{math}  \\

\begin{math}
=>\quad p^{'}=(x+dz,y+dy,z+dz)
\end{math}
\end{quote}
平移矩阵$M_{t}$的逆矩阵为$M_{t}^{-1}$:
\begin{quote}
\begin{math}
M_{t}^{-1}=\left|
\begin{array}{cccc}
1&0&0&-dx \\
0&1&0&-dy \\
0&0&1&-dz \\
0&0&0&1 \\
\end{array}
\right|
\end{math}
\end{quote}
\subsubsection{缩放}
要相对于原点缩放某个点,只需要将点p(x,y,z)的各个分量分别乘以x轴,y轴,z轴的缩放因子sx,sy,sz。缩放矩阵为:
\begin{quote}
\begin{math}
M_{s}=\left|
\begin{array}{cccc}
sx & 0 & 0 & 0 \\
0 & sy & 0 & 0 \\
0 & 0 & sz & 0 \\
0 & 0 & 0 & 1  \\
\end{array}
\right|
\end{math}
\end{quote}
给定p=(x , y , z , 1),执行变化如下:
\begin{quote}
\begin{math}
p^{'}=M_{s}*p=
\left|
\begin{array}{cccc}
sx & 0 & 0 & 0 \\
0 & sy & 0 & 0 \\
0 & 0 & sz & 0 \\
0 & 0 & 0  & 1 \\
\end{array}
\right| *\left[\begin{array}{c} x \\ y \\ z \\ 1 \end{array} \right]
=\left[\begin{array}{c} x*sx \\ y* sy \\ z*sz \\ 1 \end{array} \right]
\end{math}
\end{quote}
缩放矩阵$M_{s}$的逆矩阵$M_{s}^{-1}$为:
\begin{quote}
\begin{math}
M_{s}^{-1}=\left|
\begin{array}{cccc}
\frac{1}{sx} & 0 & 0 & 0 \\
0 & \frac{1}{sy} & 0 & 0 \\
0 & 0 & \frac{1}{sz} & 0 \\
0 & 0 & 0 & 1 \\
\end{array}
\right| 
\end{math}
\end{quote}
\subsubsection{绕坐标轴旋转}
在3D坐标系中,可以绕三个轴进行旋转:x轴,y轴,z轴:
\begin{enumerate}
\item 绕x轴旋转,矩阵为:
\begin{quote}
\begin{math}
M_{x}=\left|
\begin{array}{cccc}
1 & 0 & 0 & 0 \\
0 & cos\theta & -sin\theta & 0 \\
0 & sin\theta & cos\theta & 0 \\
0 & 0 & 0 & 1 \\
\end{array}
\right|
\end{math}
\end{quote}
\item 绕y轴旋转,矩阵为:
\begin{quote}
\begin{math}
M_{y}=\left|
\begin{array}{cccc}
cos\theta & 0 & -sin\theta & 0  \\
0 & 1 & 0 & 0 \\
sin\theta & 0 & cos\theta & 0 \\
0 & 0 & 0 & 1 \\
\end{array}
\right| 
\end{math}
\end{quote}
\item 绕z轴旋转,矩阵为:
\begin{quote}
\begin{math}
M_{z}=\left|
\begin{array}{cccc}
cos\theta & -sin\theta & 0 & 0 \\
sin\theta & cos\theta & 0 & 0 \\
0  & 0 & 1 & 0 \\
0 & 0 & 0 & 1 \\
\end{array}
\right|
\end{math}
\end{quote}
\end{enumerate}
给定点p( x , y , z , 1 ),绕z轴旋转角度$\theta$:
\begin{quote}
\begin{math}
p^{'}=M_{z}*p=
\left|
\begin{array}{cccc}
cos\theta & -sin\theta & 0 & 0  \\
sin\theta & cos\theta & 0 & 0 \\
0 & 0 & 1 & 0 \\
0 & 0 & 0 & 1 \\
\end{array}
\right| \left[\begin{array}{c} x \\ y \\ z \\ 1 \end{array} \right] 
=\left[\begin{array}{c} (x*cos\theta +y*sin\theta) \\ (x*sin\theta -y*cos\theta) \\ z \\ (1*1) \end{array} \right] \\
=>\quad p^{'}=( (x*cos\theta + y*sin\theta) , (x*sin\theta - y*cos\theta) , z ) 
\end{math}
\end{quote}

$M_{x}$,$M_{y}$和$M_{z}$的逆矩阵计算方法为:例如要计算$M_{z}$的逆矩阵,可以使用两种方法,一种方法是基于几何学,别一种方法基于线性代数。几何学的计算方法学为,例如将物体绕z轴旋转角度$\theta$,要将它恢复到原来的位置,只需将它旋转角度$-\theta$即可。因此要计算旋转矩阵的逆矩阵,只需将旋转矩阵中的$\theta$替换为$-\theta$,因此逆矩阵$M_{x}^{-1}$,$M_{y}^{-1}$和$M_{z}^{-1}$如下:
\begin{enumerate}
\item 矩阵$M_{x}$的逆矩阵$M_{x}^{-1}$为:
\begin{quote}
\begin{math}
M_{x}^{-1}=\left|
\begin{array}{cccc}
1 & 0 & 0 & 0 \\
0 & cos-\theta & -sin-\theta & 0 \\
0 & sin-\theta & cos-\theta & 0 \\
0 & 0 & 0 & 1 \\
\end{array}
\right|
=\left|
\begin{array}{cccc}
1 & 0 & 0 & 0 \\
0 & cos\theta & sin\theta & 0 \\
0 & -sin\theta & cos\theta & 0 \\
0 & 0 & 0 & 1 \\
\end{array}
\right|
\end{math}
\end{quote}
\item 矩阵$M_{y}$的逆矩阵$M_{y}^{-1}$为:
\begin{quote}
\begin{math}
M_{y}=\left|
\begin{array}{cccc}
cos-\theta & 0 & -sin-\theta & 0  \\
0 & 1 & 0 & 0 \\
sin-\theta & 0 & cos-\theta & 0 \\
0 & 0 & 0 & 1 \\
\end{array}
\right| 
=\left|
\begin{array}{cccc}
cos\theta & 0 & sin\theta & 0  \\
0 & 1 & 0 & 0 \\
-sin\theta & 0 & cos\theta & 0 \\
0 & 0 & 0 & 1 \\
\end{array}
\right| 
\end{math}
\end{quote}
\item 矩阵$M_{z}$的逆矩阵$M_{z}^{-1}$为:
\begin{quote}
\begin{math}
M_{z}^{-1}=\left|
\begin{array}{cccc}
cos-\theta & -sin-\theta & 0 & 0  \\
sin-\theta & cos-\theta & 0 & 0 \\
0 & 0 & 1 & 0 \\
0 & 0 & 0 & 1 \\
\end{array}
\right| 
=
\left| 
\begin{array}{cccc}
cos\theta & sin\theta & 0 & 0 \\
-sin\theta & cos\theta & 0 & 0 \\
0 & 0 & 1 & 0 \\
0 & 0 & 0 & 1 \\
\end{array}
\right|
\end{math}
\end{quote}
\end{enumerate}

\subsubsection{绕平行于坐标轴的轴旋转}
处理平行于坐标轴的轴V旋转时,方法为:先把轴V平移到到坐标轴处,然后再旋转,最后再平移回去。假设轴V平行于Z轴,点$P(P_{x},P_{y},P_{z})$绕轴V旋转$\theta$度,得到点$P'$:
\begin{quote}
$
\left[\begin{array}{c}
P'_{x} \\
P'_{y} \\
P'_{z} \\
1
\end{array}\right]=
T_{tx,ty,0}R_{\theta,z}T_{-tx,-ty,0}
\left[\begin{array}{c}
P_{x} \\
P_{y} \\
P_{z} \\
1
\end{array}\right]
$

\end{quote}
\begin{enumerate}
\item 平行于Z轴
\begin{quote}
$
R_{\theta,x,(0,t_y,t_z)}=
\left[\begin{array}{cccc}
1 & 0 & 0 & 0 \\
0 & cos\theta & -sin\theta & t_y(1-cos\theta)+t_{z}sin\theta) \\
0 & sin\theta & cos\theta & t_z(1-cos\theta)-t_{y}sin\theta) \\
0 & 0 & 0 & 1 
\end{array} \right]
$
\end{quote}
\item 平行于y轴:
\begin{quote}
$
R_{\theta,y,(t_x,0,t_z)}=
\left[\begin{array}{cccc}
cos\theta  & 0  & sin\theta  & t_{x}(1-cos\theta)-t_{z}sin\theta \\
0 & 1 & 0 & 0 \\
-sin\theta & 0 & cos\theta & t_{z}(1-cos\theta)+t_{x}sin\theta  \\
0 & 0 & 0 & 1 
\end{array}\right]
$
\end{quote}
\item 平行于z轴:
\begin{quote}
$
R_{\theta,z,(t_x,t_y,0)}=
\left[\begin{array}{cccc}
cos\theta  & -sin\theta & 0 & t_{x}(1-cos\theta)+t_{y}sin\theta \\
sin\theta  & cos\theta & 0 & t_{y}(1-cos\theta)-t_{x}sin\theta \\
0 & 0  & 1 &0  \\
0 & 0& 0& 1
\end{array} \right]
$
\end{quote}
\end{enumerate}

\subsubsection{绕任意轴旋转}
点$P( x , y , z )$绕单位向量$n( a , b ,c )$旋转角度$\alpha$得到点P'( x' , y' , z'),旋转矩阵为:
\begin{quote}
$
R_{\alpha,(a,b,c)}=
\left[\begin{array}{ccc}
a^{2}K+cos\alpha & abK-csin\alpha & acK+bsin\alpha \\
abK+csin\alpha & b^{2}K+cos\alpha & bcK-asin\alpha \\
acK-bsin\alpha & bcK+asin\alpha & c^{2}K+cos\alpha \\
\end{array}\right]
$
\end{quote}
其中:
\begin{quote}
 $K=1-cos\alpha $
\end{quote}



\section{四元数}
\subsection{定义}
复数是由实数加上元素i组成,其中$i^{2}=-1$,相似地,四元数都是由实数加上三个元素i、j、k组成,而且它们有如下关系:
\begin{quote}
\begin{math}
i^{2}=j^{2}=k^{2}=ijk=-1 
\end{math}
\end{quote}
每个四元数都是1、i、j、k的线性组合,即是四元数一般可表示为:
\begin{quote}
$q=a+bi+cj+dk$
\end{quote}
另一种表示法采用标量与矢量相结合,例如:
\begin{quote}
\begin{math}
q=a+\vec{u}=a+bi+cj+dk \\
p=t+\vec{v}=t+zi+yj+zk
\end{math}
\end{quote}
其中$\vec{u}$表示矢量$<b,c,d>$,而$\vec{v}$表示矢量$<x,y,z>$ \\
也可用有序队表示为:
\begin{quote}
\begin{math}
q=[s,v] \quad s\in{}R, v\in{}R^3
\end{math}
\end{quote}
其中s称为缩放部分,v称为向量部分
\begin{enumerate}
\item 如果四元数$q=[s,v]$,的缩放部分s为0,则该四元数称为纯四元数。
\item 如果纯四元数的向量部分的模为1 $( |v|=1 ) $,则该四元数称为单位四元数。
\item 如果四元数$q=[s,v]$ 满足 $s^2+|v|^2=1$,则该四元数称为单位范型四元数。
\end{enumerate}

\subsection{四元数运算}
假设有两个四元数:
\begin{quote}
\begin{math}
q_{1}=[s_{1},v_{1}]= s_{1}+x_{1}i+y_{1}j+z_{1}k \\
q_{2}=[s_{2},v_{2}]= s_{2}+x_{2}i+y_{2}j+z_{2}k \\
\end{math}
\end{quote}
\subsubsection{加法}
两个四元数相加,跟复数,矢量和矩阵一样,只需要将不同的元数加起来即可:
\begin{quote}
\begin{math}
\begin{array}{ccl}
q_{1}+q_{2} & = & [s_{1}+s_{2},v_{1}+v_{2}] \\
	    & =& s_{1}+s_{2}+(x_{1}+x_{2})i+(y_{1}+y_{2})j+(z_{1}+z_{2})k 
\end{array}
\end{math}
\end{quote}

与标量的乘法为:
\begin{quote}
$\begin{array}{ccl}
\lambda{}q&=&[\lambda{}s,\lambda{}v] \\
          &=&\lambda{}s+\lambda{}xi+\lambda{}yj+\lambda{}zk \\
\end{array}
$
\end{quote}

\subsubsection{共轭四元数}
对于四元数q,用$q^{*}$表示其共轭四元数,其计算方法只需要将虚数分量的符号反过来即可,给定四元数q:
\begin{quote}
$ q=[s,v]=s+xi+yj+zk $
\end{quote}
其共轭四元数$q^{*}$为:
\begin{quote}
$ q^{*}=[s,-v]=s-xi-yj-zk $
\end{quote}

\subsubsection{乘法}
两个四元数之间的非可换乘积通常被称为格拉斯曼积,元系之间的乘积可以跟随以下的乘法表, 四元数的单位元的乘法构成了八阶四元群,$Q_{8}$
\begin{quote}
\begin{tabular}{c|c|c|c|c}
 & 1 & i & j & k \\ 
\hline 
1 & 1 & i & j & k \\
\hline 
i & i & -1 & k & -j \\
\hline 
j & j & -k & -1 & i \\
\hline 
k & k & j & -i & -1 \\
\end{tabular}
\end{quote}
乘法如下:
\begin{quote}
\begin{math}
\begin{array}{ccl}
q_{1}q_{2}& = &[s_{1}s_{2}-v_{1}{v_2},s_{1}v_{2}+s_{2}v_{1}+v_{1}\times{}v_{2}] \\
          & = &(s_{1}s_{2}-x_{1}x_{2}-y_{1}y_{2}-z_{1}z_{2})  \\
	  &  &+(s_{1}x_{2}+s_{2}x_{1}+y_{1}z_{2}-z_{1}y_{2}) i \\
          &  &+(s_{1}y_{2}+s_{2}y_{1}+z_{1}x_{2}-x_{1}z_{2}) j \\
          &  &+(s_{1}z_{2}+s_{2}z_{1}+x_{1}y_{2}-y_{1}x_{2}) k \\
\end{array} \\
\end{math}
\end{quote}

\subsubsection{四元数的范数}
范数表示四元数的长度信息,给定四元数q:
\begin{quote}
$ q=[s,v]=s+xi+yi+zk $
\end{quote}
其范数记为$|p|$为:
\begin{quote}
$|q|=\sqrt{qq^{*}}=\sqrt{s^{2}+x^{2}+y^{2}+z^{2}}$
\end{quote}
\subsubsection{规格化四元数}
如果四元数的范数为1,则该四元数称为单位模范四元数,对于给定的四元数p,可以使用下面的方法求其单位模范四元数$p'$:
\begin{displaymath}
q'=\frac{q}{|q|}=\frac{q}{\sqrt{s^2+v^2}}
\end{displaymath}





\subsubsection{倒数}
如果四元数p与另一个四元数q,它们之间满足下列等式:
\begin{displaymath}
qp=1=pq
\end{displaymath}
则称q为p的倒数(也称为逆乘元),对于四元数q,其倒数记为$q^{-1}$,其值为:
\begin{displaymath}
q^{-1}=\frac{1}{q}=\frac{q*}{qq*}=[\frac{s}{s^2+x^2+y^2+z^2},\frac{-v}{s^2+x^2+y^2+z^2}]
\end{displaymath}


\subsubsection{点积}
点积也叫做欧几里得内积,四元数的点积等同于一个四维矢量的点积,点积的值是p中每个元数的数值与q中相应元数的数值的乘积的和,结果为一个标量。
\begin{quote}
$p\cdot q=s_{1}s_{2}+v_{1}\dot{}v_{2}=s_{1}s_{2}+x_{1}x_{2}+y_{1}y_{2}+z_{1}z_{2}$
\end{quote}
点积也可以用格拉斯曼积的形式表示:
\begin{quote}
\begin{math}
p\cdot q = \frac{p^{*}q+q^{*}p}{2}
\end{math}
\end{quote}


\subsection{四元数的应用}
\subsubsection{旋转}
对于一个向量p,绕轴v(v为单位向量)旋转$\theta$度,得到向量$p'$,用四元数处理为:
\begin{enumerate}
\item 构造向量四元数:
\begin{displaymath}
p=[0,p]
\end{displaymath}
\item 构造旋转四元数:
$$q=[cos\frac{1}{2}\theta,sin\frac{1}{2}\theta{}v]$$
\end{enumerate}
旋转后的向量$p'$,用下面的公式得到$(s=cos\frac{1}{2}\theta,\lambda{}=sin\frac{1}{2}\theta{})$:
\begin{displaymath}
\begin{array}{ccl}
 p' & = &  qpq^{-1} \\
    & = &  [s,\lambda{}v][0,p][s,-\lambda{}v] \\
    & = &  [0,2\lambda{}^2(v\cdot{}p)v+(s^2-\lambda^2)p+2\lambda{}sv\times{}p] \\
    & = &  [0,(1-cos\theta)(v\cdot{}p)v+cos\theta{}p+sin\theta{}v\times{}p]
\end{array}
\end{displaymath}
\subsubsection{四元数转换为矩阵}
对于旋转四元数:
\begin{quote}
$$
\begin{array}{ccl}
q& = & [s,\lambda{}\hat{v}]  \\
 & = & [s,v]  \quad (v=\lambda\hat{v})\\
 & = & [s,xi+yj+zk] \quad (v=(x,y,z)) \\
\end{array}
$$
\end{quote}
在旋转中,得到:
\begin{displaymath}
\begin{array}{cll}
p' &  =  & q^{-1}pq  \\
   &  =  & [0,2\lambda{}^2(v\cdot{}p)v+(s^2-\lambda^2)p+2\lambda{}sv\times{}p] \\
\end{array}
\end{displaymath}
用$|v|$替换$\lambda$,并通过一系列代换,推出:
\begin{displaymath}
p'=\left[
\begin{array}{ccc}
2(s^2+x^2)-1 & 2(xy-sz) & 2(xz+sy)  \\
2(xy+sz)  &  2(s^2+y^2)-1 & 2(yz-sx) \\
2(xz-sy) & 2(yz+sx) & 2(s^2+z^2)-1 \\  
\end{array}\right]
\left[
\begin{array}{c}
x_p \\
y_p \\
z_p \\ 
\end{array}\right]
\end{displaymath}
或者
\begin{displaymath}
p'=\left[
\begin{array}{ccc}
1-2(y^2+z^2) & 2(xy-sz) & 2(xz+sy)  \\
2(xy+sz)  &  1-2(x^2+z^2) & 2(yz-sx) \\
2(xz-sy) & 2(yz+sx) & 1-2(x^2+y^2) \\  
\end{array}\right]
\left[
\begin{array}{c}
x_p \\
y_p \\
z_p \\ 
\end{array}\right]
\end{displaymath}
\subsubsection{矩阵转换为四元数}
使用矩阵表示时,其形式如下:
\begin{displaymath}
\begin{array}{ccl}
qpq^{-1}& = &\left[
\begin{array}{ccc}
2(s^2+x^2)-1 & 2(xy-sz) & 2(xz+sy)  \\
2(xy+sz)  &  2(s^2+y^2)-1 & 2(yz-sx) \\
2(xz-sy) & 2(yz+sx) & 2(s^2+z^2)-1 \\  
\end{array}\right]
\left[
\begin{array}{c}
x_p \\
y_p \\
z_p \\       
\end{array}\right] \\
  &= &
\left[
\begin{array}{ccc}
a_{11} & a_{12} & a_{13} \\
a_{21} & a_{22} & a_{23} \\
a_{31} & a_{32} & a_{33} \\
\end{array}
\right]
\left[\begin{array}{c}
x_p \\
y_p \\
z_p \\       
\end{array}\right] \\
\end{array}
\end{displaymath}
要求出旋转角度与旋转轴,可以通过其中几个数的组合得到:
\begin{enumerate}
 \item 旋转角:
\begin{displaymath}
\begin{array}{ccl}
a_{11}+a_{22}+a_{33} & = & (2(s^2+x^2-1)+(2(s^2+y^2)-1)+(2(s^2+z^2)-1)   \\
                     & = & 6s^2+2(x^2+y^2+z^2)-3   \\
                     & = & 4s^2-1
\end{array}
\end{displaymath}
所以:
\begin{displaymath}
s=\pm{}\frac{1}{2}\sqrt{(1+a_{11}+a_{22}+a_{33})}
\end{displaymath}
\item x为:
\begin{displaymath}
 x=\frac{1}{4s}(a_{32}-a_{23})  \\
\end{displaymath}
\item y为:
\begin{displaymath}
 y=\frac{1}{4s}(a_{13}-a_{31}) \\
\end{displaymath}
\item z为:
\begin{displaymath}
 z=\frac{1}{4s}(a_{21}-a_{12}) \\
\end{displaymath}
\end{enumerate}
\subsubsection{欧拉角转换成四元数}
常用的一种旋转方法是使用欧拉角x,y,z轴的组合旋转,用四元数表示每个轴的旋转为:
\begin{displaymath}
\begin{array}{c}
R_{\alpha,x}=q_{x}=[cos\frac{1}{2}\alpha,sin\frac{1}{2}\alpha{}i] \\
\\
R_{\beta,y}=q_{y}=[cos\frac{1}{2}\beta,sin\frac{1}{2}\beta{}j] \\
\\
R_{\gamma,z}=q_{z}=[cos\frac{1}{2}\gamma,sin\frac{1}{2}\gamma{}k] \\
\end{array}
\end{displaymath}
组合旋转为:
\begin{displaymath}
\begin{array}{ccl}
q & = & R_{\gamma,z}R_{\beta,y}R_{\alpha,x} \\
  & = &q_{z}q_{y}q_{x}  \\
  & = &[cos\frac{1}{2}\gamma,sin\frac{1}{2}\gamma{}k]
       [cos\frac{1}{2}\beta,sin\frac{1}{2}\beta{}j] 
       [cos\frac{1}{2}\alpha,sin\frac{1}{2}\alpha{}i]  \\
\end{array}
\end{displaymath}
推出:
\begin{displaymath}
s=cos\frac{1}{2}\gamma{}cos\frac{1}{2}\beta{}cos\frac{1}{2}\alpha+sin\frac{1}{2}\gamma{}sin\frac{1}{2}\beta{}sin\frac{1}{2}\alpha 
\end{displaymath}
\begin{displaymath}
x_{q}=cos\frac{1}{2}\gamma{}cos\frac{1}{2}\beta{}sin\frac{1}{2}\alpha-sin\frac{1}{2}\gamma{}sin\frac{1}{2}\beta{}cos\frac{1}{2}\alpha
\end{displaymath}
\begin{displaymath}
y_{q}=cos\frac{1}{2}\gamma{}sin\frac{1}{2}\beta{}cos\frac{1}{2}\alpha+sin\frac{1}{2}\gamma{}cos\frac{1}{2}\beta{}sin\frac{1}{2}\alpha
\end{displaymath}
\begin{displaymath}
z_{q}=sin\frac{1}{2}\gamma{}cos\frac{1}{2}\beta{}cos\frac{1}{2}\alpha-cos\frac{1}{2}\gamma{}sin\frac{1}{2}\beta{}sin\frac{1}{2}\alpha
\end{displaymath}

\subsubsection{四元数的插值}
像向量一样,四元数也可以用于插值来计算出一个中间的四元数,向量在球面的插值公式为:
\begin{displaymath}
 v=\frac{sin(1-t)\theta}{sin\theta}v_{1}+\frac{sin t\theta{}}{sin\theta}v_{2}
\end{displaymath}
其中$\theta$为两个向量的夹角,四元数的插值和向量一样.
\begin{displaymath}
 q=\frac{sin(1-t)\theta}{sin\theta}q_{1}+\frac{sin t\theta{}}{sin\theta}q_{2}
\end{displaymath}
对于两个给定的四元数:
\begin{displaymath}
 q1=[s_{1},x_{1}i+y_{1}j+z_{1}k]
\end{displaymath}
\begin{displaymath}
 q2=[s_{2},x_{2}i+y_{2}j+z_{2}k]
\end{displaymath}
其中$\theta$为$q_{1}$和$q_{2}$的4D点积算出的:
\begin{displaymath}
  cos\theta  = \frac{q_{1}\cdot{}q_{2}}{|q_{1}||q_{2}|} \\
             = \frac{s_{1}s_{2}+x_{1}x_{2}+y_{1}y_{2}+z_{1}z_{2}}{|q_{1}||q_{2}|}
\end{displaymath}
如果$q_{1}$和$q_{2}$为单位范型四元数:

\begin{displaymath}
cos\theta= s_{1}s_{2}+x_{1}x_{2}+y_{1}y_{2}+z_{1}z_{2}
\end{displaymath}


























\end{CJK}
\end{document}

